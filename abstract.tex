Higher educational institutions typically purchase third-party software to manage their daily operations. Because of limited budgets, the costs to purchase some needed software is prohibitive, resulting in the purchase software that can be difficult to use, poorly developed, not fully-featured for the needs of the institution, or simply the wrong tool for the job. An alternative solution is to employ a team of student developers led by a faculty and minimal staff support to create custom, institution-specific software that meets the college's unique needs while providing students with valuable work and skill-building experience. In its sixth year of operation, our team has adapted a successful model to implement this solution. The bulk of this work is usually performed in the summer, when students are able to commit to full-time employment in the program. This case study details the student software development team model and also the experience of a team of one faculty, two staff, and seven students, and the adjustments they made in response to the COVID-19 crisis. The team has been able to continue developing software needed by the institution, while maintaining the goals of the program to engage students in a software engineering process that closely mirrors the software engineering industry.


% In the Spring 2020 semester, many academic institutions were shut down in response to the COVID-19 crisis. The shutdown has remained in place on most college campuses through the summer, negatively impacting many summer opportunities like internships and work study programs. Many programs are cancelled or have been moved to remote-only experiences for the students, diminishing the quality of the experience or removing it entirely. In its sixth year of operation, our team has been creating custom, institution-specific software that meets the college's unique needs while providing students with valuable work experience building their skills and preparing them for software engineering positions after graduation. The bulk of this work is usually performed in the summer, when students are able to commit to full-time employment in the program. This case study details the experience of a team of one faculty, two staff, and seven students, and the adjustments they made in response to the COVID-19 crisis. The team has been able to czontinue developing software needed by the institution, while maintaining the goals of the program to engage students in a software engineering process that closely mirrors the software engineering industry.
