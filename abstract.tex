Higher educational institutions, like most all large businesses, need to purchase third-party software to manage their daily operations. Because of limited budgets, the costs to purchase some needed software is prohibitive, resulting in the purchase of less apt software, or none at all. These limitations lead to interfaces that are sometimes difficult to use, poorly developed, not fully-featured for the needs of the institution, or simply the wrong tool for the job. Worse yet, contracts lock institutions into this software for multiple years, putting a strain on the faculty, staff, and students who rely on the software. The tech support by these companies is not always easily accessible and costs the institution more money. As an alternative solution, a team of student developers led by a faculty and minimal staff support can create custom, institution-specific software that meets the college's unique needs while providing students with valuable work experience building their skills and preparing them for software engineering positions after graduation. This paper highlights the successes of a five-year experiment employing college students as software developers creating software for a higher education institution.

% 183 words