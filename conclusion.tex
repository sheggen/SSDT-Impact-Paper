\section{Conclusion}
The COVID-19 crisis required many changes in our daily lives. No exception to this was the way in which we work together. Remote collaboration, particularly in the software engineering industry, will likely be a necessary part of our post-COVID-19 world. Computer Science departments across the world have known remote work to be important to our industry for years now, but rarely is any effort spent in training students to be a member of a remote workforce.

The Student Software Developers Team was forced to react to the crisis by applying our in-person, highly collaborative, pair programming-based, Agile-based framework to a remote-only environment. In the span of one month, we were able to move our entire team to a remote-only process. The first cohort of seven students, two of which had never participated in the program before, have been able to continue the work of the team without an obvious negative impact on efficiency and code quality, and despite our distance, everyone is communicating effectively with each other, including paired individuals. A second team of seven students, mostly new to the program, will begin the program in the weeks following this paper's submission, further verifying the effectiveness of our framework to a remote environment. Finally, the students are adding an additional important skill to their blossoming resumes: remote collaboration in a software engineering team.

% In conclusion, higher educational institutions, like most all large businesses, need to purchase third-party software to manage their daily operations. Because of limited budgets, the costs to purchase some needed software is prohibitive, resulting in the purchase of less apt software, or none at all. These limitations lead to interfaces that are sometimes difficult to use, poorly developed, not fully-featured for the needs of the institution, or simply the wrong tool for the job. Worse yet, contracts lock institutions into this software for multiple years, putting a strain on the faculty, staff, and students who rely on the software. As an alternative solution, a team of student developers led by a faculty and minimal staff support can create custom, institution-specific software that meets the college's unique needs while providing students with valuable work experience building their skills and preparing them for software engineering positions after graduation. This can be accomplished through a year long program with two modes of operation: Summer Internship and the Academic Year. During the Summer Internship Phase, students learn crucial skills and computer science concepts, such as frameworks, programming languages, and Agile methodologies, and work to build large new features and interfaces. Group and individual process is monitored through daily Scrum meetings and students practice pair programming to bring their designs to fruition. During the Fall and Spring terms, the team will transition to the Maintenance and Customer Support Phase. There, they will focus on smaller feature delivery and providing customer support as needed. Detractors are quick to indicate that trusting undergraduates with this task, even under the leadership of faculty or staff, is an unacceptable risk to the institution. However, the Student Software Development Team at Berea College utilizes this program to design, craft, and maintain software for a range of offices on campus. The applications have quickly became an integral part of customers' jobs, assisting in simplifying institutional management processes across campus.
