\section{Introduction}

% The growing complexity of higher education institutions means an increasing reliance on technology, in particular, software. Most institutions rely heavily on databases and management platforms, such as Banner \cite{BannerWebsite}, to manage the complexity. Similarly, learning management systems such as Moodle \cite{MoodleWebsite} and Blackboard \cite{BlackboardWebsite} improve the educational experience of students and faculty by facilitating and automating common tasks such as grading and distributing information to classes. Some of the software is open source (e.g., Moodle), which means the software itself is free, but customization, support, and maintenance can cost the institution thousands per application. According to a 2015 report about higher education spending, it is estimated that higher education institutions spend on average 4.2\% of their entire budget on information technology alone \cite{CDSBenchmarkReport}.

Though the first case of the virus was traced back to November 2019 \cite{}, it was in March 2020 that the World Health Organization (WHO) officially classified the COVID-19 outbreak as a pandemic \cite{}. Following this declaration, our institution made the decision to close its campus within the week of the WHO's announcement, one of the earliest campuses to shut down due to the COVID-19 crisis.

As academic institutions closed, their dependency on software and supporting services became more significant. The response to the virus resulted in most campus-based academic institutions migrating to a remote-based approach for many of their normal operations. This switch resulted in an influx of support requests from academic institutions, challenging software companies to keep up with these support requests and unexpected increase in usage.

COVID-19 aside, each academic institution is unique in their needs, and sometimes the tools to effectively do their job don't exist. Using existing software sometimes solves parts of the institution's needs but often is missing key features that are institution-specific. These modifications are supported by software developers through ``additional services”, usually for a cost. However, an institution's unique needs can be met by a team of students crafting custom, institution-specific software. Instead of hiring an entire software engineering team, a faculty or staff member with software engineering experience can serve as a project manager to a team of students, guiding them in the development of an application. The institution can avoid the high cost of software and maintenance, but still get applications that are tailored specifically to meet their needs.

There are a plethora of examples of students creating real-world software (i.e., software that is eventually used by the product owner to do their business) in courses \cite{tadayon2004software}, capstone experiences \cite{capstone}, and internships outside of an academic institution \cite{rochesterfirstundergradsoftwareteam}. Kaminar \cite{kaminer_2014} summarizes a few instances where students developing software that was eventually adopted by the institution. However, these adoptions occurred organically and were typically one-off ventures. To our knowledge, no examples were found of institutions hiring their students to develop custom software solutions in a multi-semester, internship model, making this program the first of its kind. Furthermore, the students are responsible for maintaining the software throughout its use at the institution.

One of the primary benefits of a program such as this is that the students gain valuable experience, preparing them for the software engineering industry after graduation. The students learn numerous technical skills and soft skills \cite{hardskills}, most of which are deemed valuable by employers for new graduates \cite{lavy2013soft}. Previous work has shown that these benefits are easily obtainable through the development of real-world applications, regardless of the setting which the application is developed (course, capstone, internship, etc.) \cite{heggen2018hiring, liu2005enriching, alzamil2005towards}.

This paper first provides context to the type of software generated by the Student Software Development Team (SSDT) by describing one of the nine systems developed over the last six years. Next, the framework for a program, which blends the best parts of a software engineering course, a capstone experience, and an external experience like an internship, is described. The paper will conclude with a discussion about how the program shifted to remote-only work during the COVID-19 crisis, exemplifying the adaptability of the program and how the students are benefitting from the framework being applied to a remote-only working environment.
