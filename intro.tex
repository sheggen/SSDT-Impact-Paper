\section{Introduction}

The growing complexity of higher education institutions means an increasing reliance on technology, in particular, software. Most institutions rely heavily on databases and management platforms, such as Banner \cite{BannerWebsite}, to manage the complexity. Similarly, learning management systems such as Moodle \cite{MoodleWebsite} and Blackboard \cite{BlackboardWebsite} improve the educational experience of students and faculty by facilitating and automating common tasks such as grading and distributing information to classes. Some of the software is open source (e.g., Moodle), which means the software itself is free, but customization, support, and maintenance can cost the institution thousands per application. According to a 2015 report about higher education spending, it is estimated that higher education institutions spend on average 4.2\% of their entire budget on information technology alone \cite{CDSBenchmarkReport}.

The dependency on software and those who provide support for it became more significant as academic institutions closed their campuses as a response to the spread of the novel coronavirus disease, shortened as COVID-19. The introduction of this virus resulted in many, if not all \cite{?}, campus-based institutions to migrate to a remote-based approach to many of their normal operations. This was similarly the case for software engineering staff, within these institutions as well as those working at companies like those mentioned above. Moreover, platforms such as these have experienced an unpredented number of support requests due to a multitude of institutions simultaneously increasing their online teaching presence. This resulted in lower or stalled productivity in a time where customers who rely on these services needed the most assistance (BIG CITE).

COVID-19 aside, each institution is unique in its needs, and sometimes the software simply doesn't exist. For example, there are nine work colleges in the United States \cite{WCCMembers, Ecclesia} that require every student to be employed while attending and student employment is considered a part of the institution's academic mission. Given there are only nine work colleges, there isn't a market for the creation of large-scale software to support work colleges despite there being a lot of need for it. Using existing software sometimes solves parts of the institutions’ needs (e.g., tracking student employment, reporting to the federal government, and evaluating student and supervisor performance) but often is missing key features that are institution-specific. These modifications are supported by software developers through “additional services”, for a cost. In the case of work colleges, which tend to be smaller institutions, the cost of modifying existing software is particularly prohibitive.

However, an institution's unique needs can be met by a team of students crafting custom, institution-specific software. Instead of hiring an entire software engineering team (which would be quite expensive), a faculty or staff member can serve as a project manager to a team of students, guiding them in the development of software. The institution can avoid the high cost of software and maintenance, but still get applications that are tailored specifically to meet their needs.

There are a plethora of examples of students creating real-world (i.e., software that is eventually used by the product owner to do their business) software in courses \cite{coursevsproject, tadayon2004software}, capstone experiences \cite{keogh2007scalable, capstone}, and internships outside of the institution \cite{rochesterfirstundergradsoftwareteam}. Kaminar \cite{kaminer_2014} summarizes a few instances where students developing software that was eventually adopted by the institution. These adoptions occurred organically and were typically one-off ventures. To our knowledge, no examples were found of institutions hiring students to develop custom software solutions in a multi-semester, internship model, making this program the first of its kind.

One of the primary benefits of a program such as this is that the students gain valuable experience, preparing them for the software engineering industry after graduation. The students learn numerous technical skills \cite{hardskills}, such as specific programming and markup languages, software engineering principles, libraries, frameworks, web development, MVC \cite{mvc}, and more. Students also learn a number of valuable soft skills, such as critical thinking, leadership, project and time management, teamwork, and problem solving; skills deemed valuable for new graduates to have by employers \cite{lavy2013soft}. Previous work has shown that these benefits are easily obtainable through the development of real-world applications, regardless of the setting with which the application is developed (course, capstone, internship, etc.) \cite{heggen2018hiring, liu2005enriching, alzamil2005towards}.

This paper describes the framework for a program which blends the best parts of a software engineering course, a capstone experience, and an external experience like an internship. The paper will further go on to explain how this program persisted during the global pandemic, COVID-19, exemplifying the adaptability that a program such as this would have at similar institutions.
