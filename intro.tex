\section{Introduction}

% The growing complexity of higher education institutions means an increasing reliance on technology, in particular, software. Most institutions rely heavily on databases and management platforms, such as Banner \cite{BannerWebsite}, to manage the complexity. Similarly, learning management systems such as Moodle \cite{MoodleWebsite} and Blackboard \cite{BlackboardWebsite} improve the educational experience of students and faculty by facilitating and automating common tasks such as grading and distributing information to classes. Some of the software is open source (e.g., Moodle), which means the software itself is free, but customization, support, and maintenance can cost the institution thousands per application. According to a 2015 report about higher education spending, it is estimated that higher education institutions spend on average 4.2\% of their entire budget on information technology alone \cite{CDSBenchmarkReport}.

As academic institutions closed their campuses in response to the spread of the novel coronavirus disease, COVID-19, their dependency on software and supporting services became more significant. The response to the virus resulted in most campus-based academic institutions to migrate to a remote-based approach for many of their normal operations. This was also the case for many software companies and their staff in the industry at-large. As many academic institutions switch to online-only operations, their reliance on software also increased, resulting in an influx of support requests from academic institutions, challenging software companies to keep up with these support requests and unexpected increase in usage.

COVID-19 aside, each academic institution is unique in their needs, and sometimes the tools to effectively do their job doesn't exist. For example, there are nine work colleges in the United States \cite{WCCMembers, Ecclesia} that require every student to be employed while attending, and their employment is usually considered a part of graduation requirements and the institution's larger mission. Given there are only nine work colleges, there isn't a large market for the creation of software to support work colleges, despite the work colleges needing the software to operate efficiently. Using existing software sometimes solves parts of the institution's needs (e.g., tracking student employment, reporting to the federal government, and evaluating student and supervisor performance) but often is missing key features that are institution-specific. These modifications are supported by software developers through ``additional services”, usually for a cost. In the case of work colleges, which tend to be smaller institutions, the cost of modifying existing software can be particularly prohibitive.

However, an institution's unique needs can be met by a team of students crafting custom, institution-specific software. Instead of hiring an entire software engineering team (which would be quite expensive), a faculty or staff member with software engineering experience can serve as a project manager to a team of students, guiding them in the development of an application. The institution can avoid the high cost of software and maintenance, but still get applications that are tailored specifically to meet their needs.

There are a plethora of examples of students creating real-world software (i.e., software that is eventually used by the product owner to do their business) in courses \cite{coursevsproject, tadayon2004software}, capstone experiences \cite{keogh2007scalable, capstone}, and internships outside of an academic institution \cite{rochesterfirstundergradsoftwareteam}. Kaminar \cite{kaminer_2014} summarizes a few instances where students developing software that was eventually adopted by the institution. However, these adoptions occurred organically and were typically one-off ventures. To our knowledge, no examples were found of institutions hiring their students to develop custom software solutions in a multi-semester, internship model, making this program the first of its kind. Furthermore, the students are responsible for maintaining the software throughout its use at the institution.

One of the primary benefits of a program such as this is that the students gain valuable experience, preparing them for the software engineering industry after graduation. The students learn numerous technical skills \cite{hardskills}, such as specific programming and markup languages, software engineering principles, libraries, frameworks, web development, MVC \cite{mvc}, and more. Students also learn a number of valuable soft skills, such as critical thinking, leadership, project and time management, teamwork, and problem solving; skills deemed valuable for new graduates to have by employers \cite{lavy2013soft}. Previous work has shown that these benefits are easily obtainable through the development of real-world applications, regardless of the setting with which the application is developed (course, capstone, internship, etc.) \cite{heggen2018hiring, liu2005enriching, alzamil2005towards}.

This paper first provides context to the type of software generated by the Student Software Development Team (SSDT) by describing one of the nine systems developed over the last six years. Next, the framework for a program, which blends the best parts of a software engineering course, a capstone experience, and an external experience like an internship, is described. The paper will conclude with a discussion about how the program shifted to remote-only work during the COVID-19 crisis, exemplifying the adaptability of the program and how the students are benefitting from the framework being applied to a remote-only working environment.
