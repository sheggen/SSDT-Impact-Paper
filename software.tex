\section{Software Developed by Students}

With every institution being unique in its software needs, it can become strenuous and expensive searching for software that meets particular needs. The challenge and experience of building entire software applications for the institution give students the opportunity to grow outside of a traditional classroom setting, while benefitting the institution with custom applications to solve their unique needs.

The software projects the software development team undertake are unique and specific to the institution. The team receives software requests from departments at the institution through direct communications and then review the requests to determine which projects are feasible and which lie outside the mission of the team. Selected projects go into a backlog to be considered in the upcoming summer. A number of factors determine which systems we will implemented, including the request’s urgency, value to the institution, and the expected effort required to complete the project. The vast majority of the applications the SSDT build are web applications, and almost all of them are built using the Python Flask framework, to reduce the number of languages and coding structures the students need to learn.

In the six years since the creation of the Student Software Development Team, coined SSDT, the team has started twelve software systems. Three of those systems are now retired and one is no longer maintained by the SSDT. Systems have been built to support the specific needs of one department, such as a tool for managing the entire Art department's collection of artifacts, while some serve the entire institution, such as the labor status forms system which is described next.

%CUT The software customers are encouraged to communicate with the SSDT at any time to request modifications or to report issues, fostering a relationship with our customers.

There is not a large market for the creation of software to support the specific needs of work colleges, considering there are only nine in the United States \cite{WCCMembers, Ecclesia}. At our institution of approximately 1,600 students, every student is required to participate in the work-study program as part of their academic requirements for graduation. Every student is required to work 10-20 hours a week, and each department at the college is allocated a specific number of students per year. The Labor Office, which regulates all student labor, was using an outdated paper-based process and spreadsheets to track all of  this information, resulting in lots of errors. This process not only required extra student labor, but also required students to walk across campus gathering signatures from multiple supervisors and academic advisors to complete their forms. A lean project estimated this process took approximately 114 minutes per student to complete a hire.

%CUTStudents are employed by the college with positions that range anywhere from cleaning the buildings, working with faculty as teaching assistants, and our team that is creating software solutions for the institution.

The SSDT was asked to develop a custom application that would replace the antiquated process. A new web application was designed and developed to be used by every labor supervisor to hire and fire students, as well as maintain and manage all other aspects of the students labor position. It was built with an administration interface, which allowed for all labor across campus to be maintained within one web application, and was equipped with all of the tools needed to manage labor in a more streamline fashion.

The new web application, with its first version completed in 2015, provided the Labor Office with a much needed improvement to their previous process. The same lean project estimated a student hire could be completed in as little as 4 minutes, with significantly less errors. The new system also opened the door for new challenges and opportunities. The web application replaced the existing process, but it introduced new processes that needed consideration. For example, sometimes employment proof is needed for a student. Now that paper forms didn't exist, a new proof of employment document had to be created from our system. For the next four years, these new features were integrated into the existing system as the customer requested them.

After five years of developing this one system, a large backlog of requests had been received and were not getting integrated. This was largely due to the language selected when we began in 2015, as well as five years of code that had drastically improved over time, but old code still lingered that occasionally didn't fit our new standards for quality. In 2019, the team decided to move the web application to a new framework that mirrored the other eight live systems also managed by the SSDT, reducing the amount of relearning necessary when switching between software systems, as well as eliminating the many mistakes made by previous programmers that still lingered.

This application began as an idea to eliminate the error prone paper-based process, and eventually evolved into an application that needed a complete overhaul due to the growing complexity of the original goal. The SSDT is driven to build software solutions for departments across the institution, with an emphasis on building maintainable software. This is why the SSDT has structured their software engineering process in a way that gives the team the necessary time needed to build applications from the ground up, while still being able to run regular maintenance on all other systems.














%%%%%%%%%% DONE: Section below moved to process.tex




%Talk about Atom Teletype, Slack video calls and screen sharing, Balsamiq paper protyping, and maintaining pair programming pedagogies, daily Scrum meetings, VPNs
















%%%%%%%%%%%%%%%%%%%%%%%%%%%%%% REALLY OLD STUFF %%%%%%%%%%%%%%%%%%%%%%%%%%%%%%%%

%\section{The Software}
% Quick paragraph... improve?
%This section begins by describing the process for selecting new requests for software, followed by a brief summary of the structure the current software that has been generated by the SSDT has been built upon. Lastly, two examples of applications built by the SSDT are presented.

%\subsection{Selecting Software Requests}
%The team receives software requests from departments and offices at the institution who are aware of the Student Software Development Team and think we can help them. In some cases, the department needs to replace old or inadequate software they already own. In other cases, they are still relying on inefficient paper processes that could easily be replaced with a software solution. It is also not unlikely for team members to reach out to departments and inquire about their needs. When a software request is received, the team reviews the request and determines which projects are feasible and which lie outside of the mission of the team. Selected projects go into a backlog to be considered in the upcoming summer. A number of factors determine which systems we will implement, including the request's urgency, value to the institution, and the expected effort required to complete the project. All of these factors are weighed in order to select the request that best fits our capabilities and current capacity.

% % CUT?
% \subsection{Software Engineering Principles}
% After a software request is selected, the summer term starts with the team following the principles and values described in the Agile Manifesto to begin building the software solution. The Manifesto does not provide concrete or descriptive instructions on how to develop software, but instead provides fundamental information to be considered throughout the entirety of the software development from project initiation to project close. Agile software development prioritizes interactions within the software team as well as with the customer, and encourages software processes that are receptive to change while still delivering quality software \cite{agilemanifesto}. % How are we using it?

% % CUT The hybrid of Scrum used by the team also intertwines the Kanban model into its practices.
% Scrum methodology is a subgroup of the Agile project management framework that articulates more details and specifications on how to employ the principles in the Manifesto within the team's software development practices with ``the goal of delivering new software capability every 2-4 weeks'' \cite{thescrumguide}. [BRIACHECKTHIS] Scrum is the most popular agile methodology; ``According to the 12th annual State of Agile report, 70 percent of software teams use Scrum or a Scrum hybrid'' \cite{}.



% Cut? The four Scrum events complemented with Kanban references these events as ``flow-based events'' to acknowledge the importance of managing the teams workflow during these events. The Summer term, when students work for 40 hours a week, is when the team can execute these events to the full extent as the team is able to convene in a way that is comparable to that of a software team that works full-time year round. The Sprint Planning begins when the team decides which customer request will be the central focus for that year. This event is used to outline the work that will be performed during the Sprint. The Development Team uses this time to forecast the range of capabilities that will be developed during the Sprint and to also set the Sprint Goal. The Sprint Goal for the SSDT is to have a beta product available by the end of the Sprint. When a software is released in beta, the majority of the software requirements have been met, however,there may be small issues that have yet to be addressed. By releasing beta versions of products, the SSDT and the Product Owner are able to observe most of the functionality of the software that has already and also test for inconsistencies. Close communication with the customers is crucial to the development of the software; staff's needs may change, college policies may update, and staff may move on to jobs elsewhere. As needs change, SSDT can adapt to and account for them.




% Repeat of above words
% \paragraph{Sprint Planning \& Paper Prototyping}
% Sprint Planning begins with analyzing the current processes that the Product Owner is utilizing*. Asking questions such as, What does the current software that is used to solve their problem look like? How does it work? Are they using any software? Where is the data that needs to be tracked currently being stored? What forms have to be filled out and which people have the authority to approve this process before it is considered to be done and put into action. For instance, if the Product Owner was requesting a software that allows students to add and remove courses to their schedules. The SSDT would need to know how students are currently able to do this task and then make sure to add it to the requirements of the software that is being built. During the SSDT's most recent Sprint, the request that was chosen involved doing an entire refactoring of a live* software.

% In order to get a better idea of the functionality of the new software should be implemented, the Development Team begins by going through multiple iterations of paper prototypes. Paper prototyping is a prototyping method in which paper is used to simulate a computer or web application. A paper prototype should hold all of the functionality that the finished user interface of the application will have; from navigation bars, drop down menus, and headers to button clicks and items that will hover on the interface. A person should be able to ``click-through'' the website via the paper prototype. To start this process, screenshots of the old software's interfaces were taken and printed out. Next these interfaces are critiqued* and analyzed to decide which parts are to be kept for the refactoring, which parts will need to be reworked, and which parts will need to be discarded altogether. The purpose of this part of the paper prototyping process is to become familiar with the current software so that the team can adequately build a better software. The team looks for things* such as bugs, broken links, slow page loading, poor user design, etc. and makes a note of these issue to ensure that they are addressed in the new software. After the individual interfaces of the previous software have been discussed and analyzed, the team divides the different interfaces into two categories: Main and Administrative. Main interfaces are the web pages that all users of the application will be able to see whilst Administrative web pages will only be accessed by system administrators, such as BC staff and SSDT's supervisor.

% Next the team breaks down the interfaces into issues in which the team will tackle* in pairs. The pair of developers then begins to design the interface that they had chosen from the interfaces that needed to be refactored. Each time a pair from the team believes that they have successfully created a valid* paper prototype, it is then tested for usability. Paper prototyping testing occurs* in the same way that a fully functional software would be tested, the person testing the paper interface will treat it as such. The tester will ``click'' on different components of the interface and the designers will physically move parts and replace parts of the paper interface in order to replicate how the real software would respond. The pair who designed the interface will take notes as the tester navigates their paper web page and when the test is complete, they use these notes to redesign and the process is repeated. Paper prototyping is done because ``bug'' and inefficiencies are easier to fix when no coding has been done yet and one can go through many iterations without having to actually having to troubleshoot actual code. The SSDT repeats this process of designing, testing, and re-designing until the entire team is satisfied with the final iteration. After all interfaces have been drafted and prototyped, the process of building the application begins and the team commences the Sprint.

%TODO Do we want to go into Flask, Peewee, Jinja, etc. or keep it generic? KEEP IT GENERIC!

%\subsection{Software Built by the Student Software Development Team}
%The vast majority of the applications created by the SSDT are browser-based web applications. The SSDT follows an MVC architectural model to structure all of the applications. The team uses Flask, a web framework that uses Python as a back-end language (i.e., the controller), designed to make the start-up of a web application quick and easy. Coupled with two Python libraries, Jinja for template rendering (i.e., the view) and Peewee for database abstraction and integration (i.e, the model), the team has all the tools needed to begin building web applications. Since most of our applications are intended to be tools for doing work (as opposed to web sites for selling products or advertising services), we leverage Bootstrap for web page styling and simplifying the implementation of the front-end of the application, particularly related to mobile-readiness and responsiveness.

%In the five years since the creation of the SSDT, we have started twelve software systems. Of those twelve, three are retired and one is not maintained by our team any longer. The remaining eight applications are regularly maintained by the SSDT. Communications with the product owners occur at a minimum twice a year, where new features are proposed by them as they discover more ways in which the software can support their work. Otherwise, customers report bugs and issues via email to the SSDT supervisor. Each system was initially designed to serve a specific need within the product owner's department. For brevity's sake, we will describe two of these applications that are of relevance to many academic institutions.

%Systems: Advancement, BCAC, BCSR, CAS, LSF, Skyz, Ulmann, Urcpp, EDDY (ret.), BART (ret.), GoIntern (ret.), Sustainability (no maintenance)


% CUT During the Sprint, the team convenes everyday in the morning in order to touch base with each of the pairs within the team and get an idea of their progress. During the Scrum meeting, the smaller teams tell what the have accomplished since the last meeting, what they will begin to work on next and what they will be completing before the next Scrum meeting, and where they are stuck or if any obstacles got in the way of completing their previous work. Scrum meeting (also called the Daily Stand-Up) is for quick communication purposes and should take no longer than 15 minutes. Sometimes small demos are done during the Scrum meeting in order to get the entire teams' input on certain features before proceeding. The programmers are expected to do a ``DDS'' in a team collaboration application called Slack, DDS stands for Did, Doing, and Stuck. During the Academic Year, these updates serve to compensate the lack of daily Scrum meetings; all students are not available at the same time while classes are in session. Team leaders examine the DDS reports to ensure progress is steady as well as evaluate who needs guidance. These updates also aid in communication between team members if they happen to not see each other. Instead of having a daily Scrum, we instead have a weekly Labor Meeting in which we present and evaluate the team's progress and state in production.

% \paragraph{SPRINT REVIEW}
% The Sprint Review's purpose is to inspect the amount of work that was completed during the Sprint and measure the amount of work needs to be completed for the next Sprint. The SSDT uses a version control application called GitHub that allows them to edit parts of the software using a copy of the application in their own environment without interfering with the main program. When the Sprint is completed, the parts of the application that were successfully finalized during the Sprint are merged into the main program. This will allow the team to then create another copy of the main program that now has the new changes and fixes to include in the next Sprint. After the Sprint is completed, the team does an overall demonstration to present the work that they have completed during the Sprint. After the team demonstration, the Product Owner is brought in so that the team can present the work that has been completed so far. It is important that the Product Owner also participates in the Review; their job is to make sure the work so far meets their criteria and requirements. They also have the authority to reject part of the application, suggest modifications to what has been built, or request a feature be added to the application. The feedback from the Product Owner during this stage is essential as it defines new criteria for the next wave of implementation.

% The Review is also used to help set goals for the next iteration. The team holds another meeting to discuss what issues and features were supposed to be completed by the end of the Sprint, but did not. Of those issues, how close are they to being completed and what is a realistic goal that be set for completion in the next Sprint. The SSDT will also review the work that was added during the Sprint or the work was removed. This will give the team insights about what goals may have been too ambitious for the Sprint deadline and what goals could have had more tasks added to them. The Review helps the SSDT to transparently assess the capabilities of the team.

% \paragraph{SPRINT RETROSPECTIVE}
% This is the final meeting after the Sprint in which the team discusses the overall process and performance of the Sprint. The Sprint Retrospective provides an opportunity for the SSDT and the Scrum Master to strategize ways to improve their methods and approaches for the next Sprint. The objective of this part of the Scrum process is not to focus so much on the work itself but to evaluate the processes of the team. During the meeting, the team finds what activities worked well and should continue to be used in the future, what went wrong during the Sprint, and what could be improved for the next iteration. This meeting is not, however, designed to assess the performance of an individual or to penalize anyone for any shortcomings. The intention of the Retrospective is to gather information about ways to improve the next Sprint. During this discussion, the team openly explores the difficulties and challenges faced during implementation. The team's strengths and triumphs are examined alongside the setbacks to better prepare for these challenges in the future.

% \subsubsection{The Academic Term}
% The Fall and Spring terms are both comprised 16 weeks* where most students work at least 10 hours a week. Upperclassmen students can work up to 15 hours a week, if permissible. In addition to class scheduling occupying the students' weekly routines, the weekly labor restrictions follow the guidelines established by the labor program at the institution.


% "MOVE ME" ``The SDDT at our institution was established over four years ago with the Student Software Development Initiative by a Computer Science faculty member.''

% "DONT THINK WE NEED THIS" ``Students learn real-world work skills such as time management, and responsibility. Some students move up into positions of leadership and build those skills. Between classes, extracurriculars, and secondary labor positions, SSDT's staff dedicate time to their labor hours.''
